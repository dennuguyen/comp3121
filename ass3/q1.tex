\usemodule[m-ass]
\usemodule[newmat]

\starttext

\subject{Question 1 by Dan Nguyen (z5206032)}

    Given a triangular grid, $T$, of non-negative integers with $n$ rows where the $i^{\text{th}}$ row has $i$ entries. The $j^{\text{th}}$ entry in row $i$ is denoted $T(i, j)$.

    A {\it route} is any path that starts at $T(1, 1)$ and travels down $T$ through diagonal pathing i.e. the next node in the {\it route} from $T(i, j)$ is $T(i + 1, j - 1)$ or $T(i + 1, j)$.

    Define {\it distance} to be the sum of all entries in the {\it route}.

    \subsubject{Part A}

    \midaligned{
    \starttable[|c|c|c|c|c|c|c|]
        \NC \NC \NC 0 \NC \NC \NC \NR
        \NC \NC 0 \NC \NC 2 \NC \NC \NR
        \NC 10 \NC \NC 0 \NC \NC 3 \NC \NR
    \stoptable
    }

    Starting from $T(1, 1) = 0$, a greedy algorithm will choose $T(2, 2) = 2$ then choose $T(3, 3) = 3$. The greedy algorithm will determine the route to be $(T(2, 2),\; T(3, 3))$ and will determine the largest {\it distance} to be 5.

    The correct {\it route} is $(T(2, 1),\; T(3, 1))$ which has the largest {\it distance} that is 10.

    \subsubject{Part B}

    Define $Q(i, j)$ as the problem of finding the largest {\it distance} of the {\it route} $T[1..i, j]$ ending with $T(i, j)$.

    Define $\text{opt}(i, j)$ as the solution to $Q(i, j)$.

    For each $1 \leq i \leq n$ and $1 \leq j \leq i$, solve for $Q(i, j)$ using dynamic programming where the recurrence is:
    \startformula \text{opt}(i, j) = \text{max}\{ \text{opt}(i - 1, j - 1) \;|\; 2 \leq j \leq i,\; \text{opt}(i - 1, j) \;|\; 1 \leq j < i \} + T(i, j)\stopformula

    The base case is $\text{opt}(1, 1) = T(1, 1)$. The order of solving $Q$ is important i.e. subproblems with lesser $i$ then lesser $j$ are solved first.

    The final answer is:
    \startformula \text{opt}(n, i) = \text{max}\{ \text{opt}(n - 1, i - 1) ,\; \text{opt}(n - 1, i)\} + T(n, i) \stopformula

    There are $n + (n - 1) + (n - 2) + ... + 1 = n(n + 1)/2$ subproblems which are solved in $O(n^{2})$. The time complexity of solving each subproblem is constant. Therefore, the overall time complexity is $O(n^{2})$ as required.

\stoptext