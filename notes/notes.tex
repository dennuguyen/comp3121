\usemodule[m-notes]
\usemodule[newmat]

\subject{Asymptotic Notation}

    Consider the following theorems.

    Thereom 1 -- Big O Notation:
    \startformula
        f(n) = O(g(n)) \text{ if } \exists \text{ } C, N > 0 \text{ such that } 0 \leq f(n) \leq Cg(n) \text{ } \forall \text{ } n \geq N
    \stopformula

    Thereom 2 -- Big $\Omega$ Notation:
    \startformula
        f(n) = \Omega(g(n)) \text{ if } \exists \text{ } c, N > 0 \text { such that } 0 \leq cg(n) \leq f(n) \text{ } \forall \text{ } n \geq N
    \stopformula

    Thereom 3 -- $\Theta$ Notation:
    \startformula
        f(n) = \Theta(g(n)) \text{ } \iff \text{ } f(n) = O(g(n)) \text{ and } f(n) = \Omega(g(n))
    \stopformula

    The general method to showing that a given $f(n)$ and $g(n)$ satisfies the above Theorems, is to rearrange the inequality with $f(n)/g(n)$ or $g(n)/f(n)$ in the middle of the inequality so it is bounded by zero on the left-hand side and some constant on the right-hand side. The Theorem satisfaction criteria are thus:
    \startitemize
        \item If $f(n)/g(n)$ converges, then it will satisfy Theorem 1. It can be said that $f(n)$ does not grow faster than $g(n)$.
        \item If $f(n)/g(n)$ does not converge, then it is divergent and does not satisfy Theorem 1. There is no need to check the type of divergence.
        \item If $g(n)/f(n)$ converges, then it will satisfy Theorem 2. It can be said that $f(n)$ does not grow slower than $g(n)$.
        \item If $g(n)/f(n)$ does not converge, then it is divergent and does not satisfy Theorem 2. There is no need to check the type of divergence.
    \stopitemize

\subject{Recurrence}

    Using recurrence to solve a problem of size $n$ reduces the problem to $a$ many subproblems of a smaller size $n/b$. The overhead cost of reducing the problem and combining the solutions of the subproblems is $f(n)$.

    The time complexity of the recurrence is given as:
    \startformula T(n) = aT\left(\frac{n}{b}\right) + f(n) \stopformula

    Reducing the $T(n/b)$ term $log_{b}a$ times yields:
    \startformula T(n) \approx n^{log_{b}a}T(1) + \sum_{i=0}^{\lfloor log_{b}n \rfloor - 1}a^{i}f\left(\frac{n}{b^{i}}\right) \stopformula

    \subsubject{Master Theorem}

        Master Theorem is used to find the time complexity of `typical' recurrence problems.

        The critical polynomial is given as:
        \startformula n^{c^{*}} = n^{log_{b}a} \stopformula

        \startitemize[n]
            \item If $f(n) = O(n^{c^{*} - \epsilon})$ for some $\epsilon > 0$, then $T(n) = \Theta(n^{c^{*}})$.
            \item If $f(n) = \Theta(n^{c^{*}})$, then $T(n) = \Theta(n^{c^{*}}log(n))$.
            \item If $f(n) = \Omega(n^{c^{*} + \epsilon})$ for some $\epsilon > 0$, and $af(n/b) \leq cf(n)$ holds for some $c < 1$ and holds $\forall$ $n > N$, then $T(n) = \Theta(f(n))$.
            \item Otherwise, Master Theorem does not apply.
        \stopitemize

        The general method to using the Master Theorem is to identify $a$, $b$, and $f(n)$; determine the critical polynomial; and setting $\epsilon$ to be some appropriate value. The Theorem satisfaction criteria are thus:
        \startitemize
            \item If time complexity of $f(n)$ is less than time complexity of critical polynomial, then case 1 applies.
            \item If time complexity of $f(n)$ is equal to time complexity of critical polynomial, then case 2 applies.
            \item If time complexity of $f(n)$ is greater than time complexity of critical polynomial and satisfies the inequality, then case 3 applies.
            \item If none of the above applies, then the time complexity of the recurrence must be determined by hand (evaluating the recurrance equation to get an expanded series then simplified).
        \stopitemize

\subject{Fast Fourier Transform} %TODO

    Inverse discrete Fourier transform (IDFT) is a sequence of coefficients of a polynomial. Discrete Fourier transform (DFT) is a sequence of values of a polynomial at the root of unity of order $m$:
    \startformula\eqalign{
        \langle A_{0}, A_{1}, ..., A_{n}, \underbrace{0, ..., 0}_{n} \rangle \xrightarrow{\text{DFT}} \langle \hat{A}_{0}, \hat{A}_{1}, ..., \hat{A}_{m - 1} \rangle \cr
        \langle B_{0}, B_{1}, ..., B_{n}, \underbrace{0, ..., 0}_{n} \rangle \xrightarrow{\text{DFT}} \langle \hat{B}_{0}, \hat{B}_{1}, ..., \hat{B}_{m - 1} \rangle \cr
        \langle C_{0}, C_{1}, ..., C_{m - 1} \rangle \xleftarrow{\text{IDFT}} \langle \hat{C}_{0}, \hat{C}_{1}, ..., \hat{C}_{m - 1} \rangle
    }\stopformula

    Fast Fourier transform (FFT) is an algorithm used to compute the DFT or IDFT. It has a time complexity of $O(nlog(n))$.

    Basically, to multiply a sequence of integers very quickly, use FFT.

\subject{Greedy Algorithm}

    Greedy algorithm solves a problem by dividing it into stages and considering the choice that is best at the time. It may not be clear if the locally optimal choice leads to a globally optimal solution. Therefore to prove correctness of our solution, consider two correctness proofs:

    \subsubject{Greedy Stays Ahead}

        Prove at every stage, there is no local solution better than the greedy algorithm's local solution.

        \startitemize[n]
            \item Label our algorithm's partial solution, $A(1..k)$, and a general partial solution, $O(1..m)$.
            \item Assume $O$ is not the same as $A$ e.g. $A$ is naturally ordered by greedy algorithm; but $O$ is optimally ordered (which optimises the measure).
            \item Define a measure, $f(a_{1}..a_{j})$ and $f(o_{1}..o_{j})$, where greedy algorithm is ahead of $O$ e.g. $f$ is last completion of first $j$ elements; total weight of first $j$ elements; or largest index of first $j$ elements.
            \item Define a goal to compare measures e.g. $f(a_{1}..a_{j}) \leq f(o_{1}..o_{j})$; or $f(a_{1}..a_{j}) \geq f(o_{1}..o_{j})$ $\forall$ $j \leq k$.
            \item Prove the above goal by induction.
            \item $A(1..j)$ stays ahead of $O(1..j)$ with respect to $f(1..j)$, therefore $A(1..j)$ is optimal.
        \stopitemize

    \subsubject{Exchange Argument}

        Consider a correct algorithm to the solution and morph it into the greedy solution without making any worse.

        \startitemize[n]
            \item Label our algorithm's solution, $A$, and general solution, $O$.
            \item Assume $O$ is not the same as $A$ e.g. there is an element of $O$ not in $A$ and an element of $A$ not in $O$; or there are 2 consecutive elements in $O$ in different order than in $A$ (inversion).
            \item Explain how $A$ and $O$ can be different.
            \item Swap elements in $O$ and argue that $O$ is no worse than $A$.
            \item Argue that we can continue to swap and eliminate all differences between $O$ and $A$ without worsening $O$.
            \item Therefore $A$ is just as good as $O$ and is therefore optimal itself.
        \stopitemize

\subject{Kruskal's Algorithm} %TODO

    Outputs a minimum spanning tree (MST) for a time complexity of $O(|E|log(|V|))$.

\subject{Dynamic Programming} %TODO

    Dynamic programming solves a problem by recursively solving overlapping subproblems with optimal subsolutions which are combined into an optimal solution for the full problem. Dynamic programming problems consist of defining the subproblem, recurrence relation, and base cases. The general method to solving these problems is:
    \startitemize[n]
        \item For $1 \leq i \leq n$, let $P(i)$ be the problem of determining $opt(i)$.
        \item Define $opt(i)$ in context to the problem.
        \item Define the recurrence relation: $opt(i) = min/max(opt(?), opt(?))$.
        \item Define base cases e.g. $opt(1) = ?$.
        \item Identify the order of which subproblems should be solved if appropriate e.g. $i$ must be solved in descending order.
        \item State the final solution e.g. $max_{1 \leq i \leq n}(opt(i))$.
        \item Determine the time complexity e.g. there are $O(n)$ subproblems, each solved in $O(n)$ time, therefore the overall time complexity is $O(n^{2})$.
    \stopitemize

    It is possible to have multiple iterators, if more information needs to be passed recursively.

\subject{Max Flow}

    Max flow problems consist of a flow network construction then application of some max flow min cut algorithm.

    \subsubject{Flow Network Construction}

        Flow networks are directed graphs which have {\it capacity constraint} and {\it flow conservation}. To fully construct a flow network, do the following:

        \startitemize[n]
            \item Let there be a flow network represented as a directed graph.
            \item Define the flow.
            \item Define the nodes.
            \item Define the node capacities.
            \item Define the edges.
            \item Define the edge capacities.
            \item Define the source.
            \item Define the sink.
        \stopitemize

    \subsubject{Max Flow Min Cut}

        The max flow of a flow network is equal to the min cut's capacity (cutting across a set of edges which bisects the flow network). To find the max flow, do the following:

        \startitemize[n]
            \item Define the max flow, $f$.
            \item Apply Ford-Fulkerson, $O(|E|f)$, or Edmonds-Karp, $O(min(|V||E|^{2}, |E|f))$, algorithm to compute $f$.
            \item Explain how $f$ is used.
        \stopitemize

    \subsubject{Max Bipartite Matching}

        Max bipartite matching problem can be identified from a flow network problem with two disjoint sets of nodes. Convert the max bipartite matching problem to a flow network problem by modifying the flow network construction with the following:
        \startitemize[n]
            \item Replace the source with a super-source which is an arbitrary coordinate to a set of nodes.
            \item Define the capacity of edges to super-source.
            \item Replace the sink with a super-sink which is an arbitrary coordinate to a set of nodes.
            \item Define the capacity of edges to super-sink.
        \stopitemize

        Typically, capacities of edges to super-sources and super-sinks are $1$.

        The time complexities for both Ford-Fulkerson and Edmonds-Karp are $O(|E||V|)$.

\subject{String Matching} %TODO

    String matching is determining if a given string, $A$, has a contiguous given substring/pattern, $B$.

    \subsubject{Rabin-Karp Algorithm}

        Rabin-Karp algorithm uses hashing and recursion to efficiently find matching strings in average $O(n + m)$ but is limited by unlikely possibility of hash collisions, therefore cannot guarantee worst case performance, $O(nm)$.

        \startitemize[n]
            \item Map symbol $s_{i} \rightarrow i$: $S = \{s_{0}, s_{1}, ..., s_{d-1}\} \rightarrow \{0, 1, ..., d-1\}$.
            \item Construct and evaluate hash integer using Horner's rule: $H(B) = h(b_{0}b_{1}...b_{m}) = d^{m-1}b_{0} + d^{m-2}b_{1} + ... + db_{m-2} + b_{m-1}$.
            \item Choose large prime number: $p$.
            \item Let $H(B) = h(B) \% p$.
            \item Compute hash, $H(A_{s})$, for each $A_{s} = a_{s}a_{s+1}...a_{s+m-1}$ of $A$. Using recursion to compute hash: $H(A_{s+1}) = dH(A_{s}) - (d^{m}a_{s} + a_{s + m}) \% p$.
            \item Compare hash values of $H(B)$ and $H(A_{s})$. Do symbol-by-symbol matching $\iff$ $H(B) = H(A_{s})$.
        \stopitemize

    \subsubject{Finite Automata} %TODO

    Pattern $B$ can be converted into a finite automata (simple state machine). The method (by-hand) to get the state transition table from $B$ is:
    \startitemize[n]
        \item Draw out a table with rows for each state (i.e. each index in $B$); and columns for each transition (i.e. each symbol in $B$).
        \item Input the ???.
    \stopitemize

    The method (by-hand) to get the state transition diagram from the table is:
    \startitemize[n]
        \item Draw nodes for each row.
        \item Draw directed arrows from the row's index to the index specified in the transition column and label it with the transition symbol.
    \stopitemize

    \subsubject{Knuth-Morris-Pratt Algorithm} %TODO

    Knuth-Morris-Pratt algorithm uses finite automata and recursion to efficiently find matching strings in $O(n)$.
    \startitemize[n]
        \item Suppose $B_{s} = b_{0}..b_{s-1}$ is longer prefix of $B$ which is suffix of $A_{i} = a_{0}..a_{i - 1}$.
        \item If $a_{i} = b_{s}$ then let $s = s + 1$; else, recursively check condition by letting $s = \pi(s)$.
        \item If $s = m$ then string match found.
    \stopitemize

\subject{Linear Programming} %TODO

    Linear programming optimises linear functions subject to various conditions. Standard form of the primal linear program, $P$, is: maximise $z({\bf x}) = {\bf c}^T {\bf x}$, subject to $A{\bf x} \leq {\bf b}$ and ${\bf x} \geq {\bf 0}$.

    The standard form of the dual linear program, $P^{*}$, is: minimise $z^{*}({\bf y}) = {\bf b}^T {\bf y}$ subject to $A{\bf y} \geq {\bf c}$ and ${\bf y} \geq {\bf 0}$.


    \subsubject{Weak Duality Theorem}

    If $x = \langle x_{1}, ..., x_{n} \rangle$ is any feasible solution for $P$ and $y = \langle y_{1}, ..., y_{m} \rangle$ is any feasible solution for $P^{*}$ then:
        \startformula
            z(x) = \sum_{j=1}^{n}c_{j}x_{j} \leq \sum_{i=1}^{n}b_{i}y_{i} = z^{*}(y)
        \stopformula

        Any $x$ is a lower bound for $P^{*}$ and any $y$ is an upper bound for $P$, so if $x = y$ then optimal solution has been found.

    \subsubject{Integer Linear Programming}

    Integer linear programming restricts some or all variables to be integers. standard form is: maximise ${\bf c}^T {\bf x}$ subject to $A{\bf x} + {\bf s} = {\bf b}$, ${\bf s} \geq {\bf 0}$, and ${\bf x} \geq {\bf 0} \in \mathbb{Z}^{n}$.

\subject{Intractability} %TODO