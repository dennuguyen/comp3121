\usemodule[m-ass1]
\usemodule[newmat]

\starttext

\subject{Question 1 by Dan Nguyen (z5206032)}

    An array, $A$, of size $n$ has only positive integers i.e. $A[i] > 0$ for $i \in \mathbb{Z}$. Let there be a pair of indices $i$ and $j$ where $i < j$. These indices are {\it consistent} if $A[j] - A[i] = j - i$.

    Rearrange the {\it consistency} rule as:
    \placeformula[eqn:q1_consistent_rule]
    \startformula A[i] - i = A[j] - j \stopformula

    Consider the index, $i$, which has the range from $1$ to $n$ inclusive. This index will be used to iterate over $A$.

\subsubject{Part A}

    Let there be a hash table, $H$, of an appropriate size larger than or equal to $n$. $H$ is initially empty.

    To count {\it consistent} indices, $A[i] - i$ is looked-up in $H$ for each $i$ in $A$. If the look-up was successful, then the counter for discovered pairs of {\it consistent} indices is incremented. Otherwise, the value, $A[i] - i$, is inserted into $H$. Iteration over $A$ has an expected time-complexity of $O(n)$, and hash table look-ups and insertions have an expected time-complexity of $O(1)$ - giving a final expected time-complexity of $O(n)$.

\subsubject{Part B}

    Let there be an AVL tree, $B$, of size $n$ which stores the value, $A[i] - i$, for each $i$ in $A$. Iteration through $A$ has time-complexity $O(n)$ and an AVL tree insertion has a worst time-complexity of $O(log(n))$. This gives a final worst time-complexity of $O(log(n))$.

    After iterating through $A$, merge-sort $B$ so that the values of $B$ are ordered. Merge-sort has a worst time-complexity of $O(nlog(n))$.

    To count {\it consistent} indices, a binary search in $B$ is done to find a value equal to $A[i] - i$ for each $i$ in $A$. A successful binary search increments the counter for discovered pairs of {\it consistent} indices. Otherwise, iteration over $A$ is continued. Iteration over $A$ has a time-complexity of $O(n)$ and a binary search through an AVL tree has a worst time-complexity of $O(log(n))$ - giving a final worst time-complexity of $O(log(n))$.

\stoptext